%% Run LaTeX on this file several times to get Table of Contents,
%% cross-references, and citations.

\documentclass[11pt]{book}
\usepackage{gvv-book}
\usepackage{gvv}
%\usepackage{Wiley-AuthoringTemplate}
\usepackage[sectionbib,authoryear]{natbib}% for name-date citation comment the below line
%\usepackage[sectionbib,numbers]{natbib}% for numbered citation comment the above line

%%********************************************************************%%
%%       How many levels of section head would you like numbered?     %%
%% 0= no section numbers, 1= section, 2= section, 3= subsection %%
\setcounter{secnumdepth}{3}
%%********************************************************************%%
%%**********************************************************************%%
%%     How many levels of section head would you like to appear in the  %%
%%				Table of Contents?			%%
%% 0= chapter, 1= section, 2= section, 3= subsection titles.	%%
\setcounter{tocdepth}{2}
%%**********************************************************************%%

%\includeonly{ch01}
\makeindex

\begin{document}

\frontmatter
%%%%%%%%%%%%%%%%%%%%%%%%%%%%%%%%%%%%%%%%%%%%%%%%%%%%%%%%%%%%%%%%
%% Title Pages
%% Wiley will provide title and copyright page, but you can make
%% your own titlepages if you'd like anyway
%% Setting up title pages, type in the appropriate names here:

\booktitle{Geometry}

\subtitle{Through Algebra}

\AuAff{G. V. V. Sharma}


\titlepage


\begin{copyrightpage}{2022}
%Title, etc
\end{copyrightpage}

\tableofcontents


\setcounter{page}{1}

\begin{introduction}
This book shows how to solve problems in geometry using trigonometry and coordinate geometry. 

\end{introduction}

\mainmatter

\chapter{Triangle}
Consider a triangle with vertices
		\begin{align}
			\label{eq:tri-pts}
			\vec{A} = \myvec{1 \\ -1},\,
			\vec{B} = \myvec{-4 \\ 6},\,
			\vec{C} = \myvec{-3 \\ -5}
		\end{align}
\section{Vectors}
\input{chapters/triangle/exam}
\section{Median}
\input{chapters/triangle/median}
\section{Altitude}
\input{chapters/triangle/altitude}
\section{Perpendicular Bisector}
\input{chapters/triangle/perp-bisect}
\section{Angle Bisector}
\input{chapters/triangle/angle-bisect}
%

\end{document}

 